\section*{Esercizi di informatica assegnati il 09/01/2019}

\subsection*{Documentazione}

Crea la documentazione e i manuali utente. \subsection*{Lavoro atteso}

Manuale per programmatori e utenti. \subsection*{Relazioni e documentazioni}

Creare una relazione con le informazioni utili a descrivere il lavoro fatto e le informazioni per gli altri utenti che devono utilizzare tale lavoro. \subsection*{Svolgimento}

Vedi relazione su altro foglio \subsection*{Fonti}

Appunti presi in classe Questo documento\+: relazione\+\_\+info\+\_\+20190109\+\_\+ss.\+fabrizi.\+odt Percorso\+: \+\_\+/home/user/\+Desktop/mondo/\+\_\+ \subsection*{Allegati}

Nessuno

\section*{R\+E\+L\+A\+Z\+I\+O\+NE I\+N\+F\+O\+R\+M\+A\+T\+I\+CA}

\subsection*{\+\_\+“\+Casa Discografica”\+\_\+}

\subsection*{G\+R\+U\+P\+PO}

Herman Hausherr e Simone Fabrizi; \subsection*{O\+G\+G\+E\+T\+TO}

Creare un manualre per i programmatori e un manuale per gli utenti; \subsection*{M\+A\+T\+E\+R\+I\+A\+LI U\+T\+I\+L\+I\+Z\+Z\+A\+TI}

Pc, monitor, tastiera e mouse; \subsection*{P\+R\+O\+G\+R\+A\+M\+MI U\+T\+I\+L\+I\+Z\+Z\+A\+TI}

Doxygen, Libreoffice e Mozilla; \subsection*{S\+V\+O\+L\+G\+I\+M\+E\+N\+TO}


\begin{DoxyEnumerate}
\item Abbiamo analizzato la consegna evidenziando le parole chiave;
\item Scaricato tutti i programmi necessari (doxygen);
\item Guardato come funziona quest’ultimo programma in quanto nuovo;
\item Fatto delle prove per capire il suo funzionamento; \subsection*{A\+P\+P\+R\+O\+F\+O\+N\+D\+I\+M\+E\+N\+TI}
\end{DoxyEnumerate}

Doxygen = Doxygen è lo strumento standard di fatto per generare documentazione da sorgenti C ++ annotate, ma supporta anche altri linguaggi di programmazione popolari come C, Objective-\/C, C \#, P\+HP, Java, Python, ecc. \subsection*{F\+U\+N\+Z\+I\+O\+N\+A\+M\+E\+N\+TO D\+O\+X\+Y\+G\+EN}


\begin{DoxyItemize}
\item Scaricare il programma dal terminale (apt-\/get install doxygen);
\item Creare una cartella di lavoro dove eseguire i comandi;
\item Creare file di configurazione (doxygen -\/g Doxyfile);
\item Modificare il file di configurazione in base alle proprie esigenze;
\item Eseguire il file modificato (doxygen Doxyfile);
\item Verrà creata una cartella html nella quale ci sarà il file da aprire (index.\+html);
\item Nella cartella vanno messi i file desiderati, ad esempio Php e Markdown sicché il programma doxygen possa leggere il file. 
\end{DoxyItemize}