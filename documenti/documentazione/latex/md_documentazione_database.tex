\section*{Creazione Database}

Prima cosa ho creato il database con le istruzoni che ci sono state date in classe. Poi ho creato 2 account con permessi diversi.

\section*{Creazione utenti}

L\textquotesingle{}utente admin ha tutti i permessi e una sua correspettiva password.\+L\textquotesingle{}utente user invece ha solo alcuni permessi cioè la S\+E\+L\+E\+CT. Per dare tutti i permessi all\textquotesingle{}amdin ho utilizzato\+: \begin{quote}
G\+R\+A\+NT A\+LL P\+R\+I\+V\+I\+L\+E\+G\+ES ON documentazione.$\ast$ TO admin@\textquotesingle{}\textquotesingle{} I\+D\+E\+N\+T\+I\+F\+I\+ED BY \textquotesingle{}la password\textquotesingle{}. \end{quote}


Per dare solo il permesso select all\textquotesingle{}user ho usato \begin{quote}
G\+R\+A\+NT S\+E\+L\+E\+CT ON documentazione.$\ast$ to user . \end{quote}


Così facendo ho creato due utenti con permessi diversi.

\% any valid host

\section*{Apertura accesso al database da host esterni}

Per far si che un host possa accedere al database bisogna cambiare una configurazione nel file 50-\/server.\+cnf che nella parrot si trova nella directory /etc/mysql/mariadb.conf.\+d bisogna cambiare l\textquotesingle{}ip in \char`\"{}bind-\/address\char`\"{} e per far si che tutti possano accedervi al posto dell\textquotesingle{}ip bisogna mettere \char`\"{}$\ast$\char`\"{}.

\section*{Accesso al database}

Chi volesse accedere al database deve utilizzare (da una rete uguale a quella del server) il comando "mysql -\/h (indirizzo ip del server che può essere modificato nel file hosts che si trova nella cartella /etc e assegnare all\textquotesingle{}indirizzo un nome in modo tale da non dover inserire ogni volta l\textquotesingle{}ip del server) -\/u user -\/ppasswd documentazione. la password l\textquotesingle{}ho resa pubblica perchè l\textquotesingle{}utente user può fare solo la select. S\+O\+LO chi ha bisogno dell\textquotesingle{}account admin verrà data la password. \section*{Creazione rete privata}

Abbiamo creato una rete privata. Per farlo abbiamo usato il comando \char`\"{}ifconfig eth0\+:1 192.\+168.\+21.\+1\char`\"{}. Questo indirizzo ip è del server. Tutti quelli che vogliono far parte della rete devono usare il comando \textquotesingle{}ifconfig eth0\+:1 192.\+168.\+21.(100+numero di registro)\textquotesingle{}. 