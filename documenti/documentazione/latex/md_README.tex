\section*{Clonare il repository}

Da fare solo una volta. 
\begin{DoxyCode}
$ git clone https://github.com/simonemastella/5BIT.git
\end{DoxyCode}
 \section*{Aggiornamento}

Si chiede algi alunni di aggiungere informazioni sui lavori svolti e sugli incarichi ricevuti.

\section*{Docente\+: Roberto Picchiottino}

Sono il docente di riferimento.

\subsection*{Alcuni consigli}

Leggere e produrre documentazione, l\textquotesingle{}alunno (gli alunni) che curano la documentazione avranno cura di informare come procedere nel migiore dei modi. Io inizio con md. Attendo fiducioso di sapere quali strategie individuate per documentare il progetto. \+:boom\+:

\subsubsection*{Copia ...}

Strano! Ti consiglio di copiare! Non è una svista!

Quando trovi una soluzione che ti piace, cerca di capire come è stata implememntata e utilizza il codice che la mette in pratica.

Fai molta attenzione alla licenza d\textquotesingle{}uso\+: deve essere compatibile con quella che utilizzi.

\subsubsection*{Aggiorna il repository}

Alcuni comandi utili ... 
\begin{DoxyCode}
git add .
git commit -m "Messaggio con indicazioni del commit"
git status
git pull
git push origin master
\end{DoxyCode}


\subsubsection*{U\+RL utili}


\begin{DoxyItemize}
\item \href{https://guides.github.com/features/mastering-markdown/}{\tt Guida a Mark\+Down}
\item \href{https://help.github.com/articles/basic-writing-and-formatting-syntax/}{\tt Sintassi di base}
\end{DoxyItemize}

\section*{Alunno 1}

Adamo Elia lavoro per il database (n°1) {\bfseries V\+I\+VA W\+I\+N\+D\+O\+WS} \section*{Alunno 2}

Prova {\bfseries U\+SO V\+I\+S\+U\+AL S\+T\+U\+D\+IO} \section*{Alunno 3}

Ciao sono Bionaz. \section*{Alunno 4}

Cecilia Corbara, 11 DB e form \section*{Alunno 5}

Ciao sono Edoardo Di Vita, 6 git master \section*{Alunno 6}

Ciao, sono Fabrizi e lavoro per la documentazione (n°7) \#$\ast$$\ast$\+F\+O\+R\+ZA R\+O\+M\+A$\ast$$\ast$ \subsection*{$\ast$$\ast$\+L\+A\+Z\+IO M\+E\+R\+D\+A$\ast$$\ast$}

\section*{Alunno 7}

Ciao sono {\itshape Gallo} approvo il commento di Gyppaz e lavoro sulla documentazione 5. \section*{Alunno 8}

Sono Guerini,il mio lavoro è connettere il Db a libreoffice. Sono Guerini,il mio lavoro è il numero 9. L\textquotesingle{}obbiettivo è quello di connettere il Db a libreofffice. Nella prima parte del compito era richiesto di creare un collegamento al database. Per creare il collegamento con libreoffice base era necessario avere un database. Dopo aver creato su maria\+DB il database chiamato “discografia” con annessa password $\ast$$\ast$$\ast$$\ast$$\ast$$\ast$$\ast$$\ast$$\ast$$\ast$$\ast$$\ast$$\ast$, ho creato le tabelle con il codice proposto dall’esercizio su ferret dato dal professore. Una volta controllato il procedimento per la giusta creazione del database, ho fatto il collegamento con libreoffice base. È stato necessario eseguire tutti i passaggi su base per riuscire ad avere una connessione riuscita. Dopodichè ho creato tutte le relazioni su base ed ho fatto la parte grafica base del database \section*{Alunno 9}

Ciao sono {\itshape Gyppaz} approvo il commento di Gallo e lavoro sulla documentazione 5. \section*{Alunno 10}

Ciao, mi chiamo {\bfseries Hermann Hausherr} \section*{Alunno 11}

Ciao sono {\bfseries Gabriele Marchesano}, sono {\bfseries 3 Web master} \section*{Alunno 12}

Ciao, sono l\textquotesingle{}alunno {\bfseries Simone Mastella}. Per questo lavoro ho avuto il compito di numero 6, cioè di gestire git e una sua repository condivisa.

\href{https://gist.github.com/simonemastella/1792e8dd3cc8a8878825a4d2df676300}{\tt Come inizializare una repository}

\href{https://gist.github.com/simonemastella/90364ec267d65a5328bd97c23aee1864}{\tt Quali comandi usare per un uso ordinario}

\href{https://gist.github.com/simonemastella/e69cd0a0d9fe151e52c0fc53ac4ad3ea}{\tt Come creare un .gitignore}

\href{https://gist.github.com/simonemastella/2ad4f08ed6f18af7e102a8e1a573dd8d}{\tt Come creare la tua G\+PG Key}

\href{https://gist.github.com/simonemastella/3e763531b32e1db583e2dcb4fdd668a8}{\tt Quali comandi usare per avere i commit verificati}

\href{https://gist.github.com/simonemastella/ecd089c6106a961eb9272a40c5b16d5a}{\tt Salvare le credenziali} \section*{Alunno 13}

Christopher \section*{Alunno 14}

Ciao, sono Petrocca Fabio detto $\ast$$\ast$\+\_\+\+Fabenz\+\_\+$\ast$$\ast$. \section*{Alunno 15}

Ciao sono Edith Piffari, lavoro per il {\bfseries 4 P\+HP database} \section*{Alunno 16}

Giacomo Raffa. Io lavoro sull\textquotesingle{}esercizio 1. devo creare il database e il server \subsection*{Casa discografica}

Avevo creato il database e due utenti con permessi diversi. Ho cambiato indirizzo ip nel file 50-\/server.\+cnf all\textquotesingle{}interno della directory /etc/mysql/mariadb.conf.\+d in modo che qualcuno possa accedere al database da remoto.

\section*{Alunno 17}

Ciao sono {\itshape Scarpante}, ho svolto la parte numero 2 ({\itshape D\+BA} {\itshape tabella}) insieme a Vergaro \section*{Alunno 18}

Sposato Fabio è un figo, lavoro 8 grafico. \section*{Alunno 19 Taut Denisa}

Ciao sono {\itshape Taut} approvo il commento di Gyppaz e Gallo Mi è stato assegnato il compito 9 DB \section*{Alunno 20}

\section*{Alunno 21}

ciao sono Jacopo, ho svolto la parte numero 2 ({\itshape D\+BA} {\itshape tabella}) insieme a {\itshape Scarpante}

\section*{Alunno 22 (Il secchione)}

Ciao, sono l\textquotesingle{}alunno preferito dal proff. Faccio tutto quello che mi chiede e provo a migliorarmi di continuo. Per questo lavoro ho avuto il compito di rompere le scatole a tutti i miei compagni.

\subsection*{DB Admin\+: mariadb.}

Devo controllare che diano ai compagni un nome host da inserire in /etc/hosts e i dati per accedere al db sia in modalità ristretta che in modalità amministratore. Dovrà anche documentare e spiegare come va configurato mariadb per accettare connessioni da rete diversa da localhost.

\subsection*{D\+BA Table}

Mi devo assicurare che i compagni che hanno da creare il database creino e pubblichino un file con i comandi per creare il database. Si parte dal \href{./manuali/documenti.ger}{\tt file ferret} che il docente ha messo a disposizione ma vanno aggiunti i campi opportuni e gli autoincrement per le chiavi virtuali.

\subsection*{Web master}

Gli alunni in questo reparto devono\+: immaginare il sito internet, assicurarmi che ci siano gli elementi di base, decidere quali framework utilizzare. Fare qualche esempio di pagina e collaborare con chi crea le funzioni php per il sito al fine di avere un lavoro omogeneo.

\subsection*{Php database}

I compagni che hanno questo compito devono lavorare al meglio in quanto senza il loro apporto non possiamo collegarci al database.

\subsection*{G\+IT Master}

Da lui/loro ci aspettiamo grandi cose\+: gestione del git, come creare il file .gitignore al fine di non divulgare info personali, documentazione su come migliorare la gestione del git.

\subsection*{Documentazione}

L\textquotesingle{}idea consiste nel creare la documentazione in automatico, i compagni che hanno questo compito devono dirci come inserire i commenti nei file al fine di avere una gestione dei documenti automatizzata. Pare stiano lavorando con \href{http://doxygen.nl/}{\tt Doxy\+Gen} che si installa anche su debian in modo semplice. Spero proprio ci dicano quali sono i comandi migliori da dare e come creare il file di configurazione per la generazione della documentazione.

\subsection*{Grafico}

Da loro devo avere i file del logo, la spiegazione sui colori scelti e i vari formati per i differenti scopi. Mi aspetto anche un modello di carta intestata e, soprattutto, le informazioni per reperire questi documenti.

\subsection*{DB Libreoffice}

Si devono collegare al database con libreoffice e creare le maschere per inserire i dati e per vedere i risultati. Il prodotto finale sarà un sito internet ma potrebbe risultare utile avere un accesso con libreoffice.

\subsection*{DB, Dati e php}

Come utilizzare A\+J\+AX per migliorare la fruibilità del sito.

\subsection*{DB Form}

Le form per inserimento dati sono piuttosto complicate, questi alunni dovranno iniziare a lavorare su questo aspetto. Da notare che le operazioni da farsi sui dati, oltre alla S\+E\+L\+E\+CT, sono I\+N\+S\+E\+RT, U\+P\+D\+A\+TE e D\+E\+L\+E\+TE.

\subsection*{G\+A\+N\+TT e coordinamento}

Questo ruolo dovrebbe essere simile al mio. Deve, come me, essere a conoscenza di chi fa cosa e deve controllare che lo faccia e relazioni.

\subsection*{....}

E via dicendo .... state attenti che vi controllo. 